\section*{Network spandrels reflect ecological assembly \citep{Maynard2018} - \textit{Intro  and Discussion}}

Ecological networks that exhibit stable dynamics should theoretically persist longer than those that fluctuate wildly. Thus, network structures which are over-represented in natural systems are often hypothesised to be either a cause or consequence of ecological stability. Rarely considered, however, is that these network structures can also be by-products of the processes that determine how new species attempt to join the community. Using a simulation approach in tandem with key results from random matrix theory, we illustrate how historical assembly mechanisms alter the structure of ecological networks. We demonstrate that different community assembly scenarios can lead to the emergence of structures that are often interpreted as evidence of 'selection for stability'. However, by controlling for the underlying selection pressures, we show that these assembly artefacts or spandrels are completely unrelated to stability or selection, and are instead by-products of how new species are introduced into the system. We propose that these network-assembly spandrels are critically overlooked aspects of network theory and stability analysis, and we illustrate how a failure to adequately account for historical assembly can lead to incorrect inference about the causes and consequences of ecological stability.

\underline{Introduction}
\begin{itemize}
	\item Ecological networks tend to be strikingly non-random
	\begin{itemize}
		\item It is hypothesised that this occurs because selection 'prunes' unstable configurations, resulting in stable patterns
		\item for example wildly fluctuating networks would not be expected to persist through time die to stochastic extinctions
	\end{itemize}
	\item but it could be that these structures are just 'artefacts of assembly' with no inherent connection to stability
	\item Assembly of biological systems is dictated by two forces
	\begin{enumerate}
		\item How, when and why variation is introduced into a system.
		\item Selective mechanisms which determine what features persist at what frequencies. 
	\end{enumerate}
	\item While selection is a dominant force we must not forget assembly constraints!
	\begin{itemize}
		\item It is possible for features which appear to have current utility to have been occurred as a by-product of the way the system was formed and have no adaptive origin.
		\item dubbed the 'network spandrel' - a nod to cathedral archways which appear to have been selected but are in fact a by-product of construction.
	\end{itemize}
	\item A 'network spandrel' refers to any network property which emerges as a by-product of how the species joins the community.
	\item Disentangling whether an empirical pattern has emerged due to assembly constraints or selection is very difficult. Requires experimental, observational and theoretical evidence.
\end{itemize}

\underline{Assembling Ecological Communities}
\begin{itemize}
	\item Again they use the Lokta-Volterra model.
	\item They set the parameters to reach a steady state
	\item After each equilibrium is reached they add a new species to the simulated community, then run to equilibrium again. This new species may or may not establish, and may or may not lead to one or more extinctions of other species.
	\item some stuff about methods, summarised well in fig1.
	\item lots of results which are interesting but difficult to summarise
\end{itemize}

\underline{Discussion}
\begin{itemize}
	\item Biological systems are not only the result of selection! But also how variation is introduced into systems!
	\item This study shows that different assembly processes leave different fingerprints on the resulting network giving the appearance of different selective pressures.
	\item Spandrels reflect the historic processes which shap	ed a system (while not specifically saying anything about selection)
	\item not accounting for spandrels leads to incorrect inferences about selection and stability.
\end{itemize}