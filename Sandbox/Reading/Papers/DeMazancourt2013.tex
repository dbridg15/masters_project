\section*{Predicting ecosystem stability from community composition and biodiversity \citep{DeMazancourt2013} - \textit{Intro and Discussion}}
Second, is it really useful to lump together everyone born between 1984 and 1993 into one group? I was born in 1989 (yes, yes, I am peak millennial, I came out of the womb holding an avocado trying to get a discount on a train ticket). I tried to talk to a colleague born in 1993 recently about Snapchat, and before she finished explaining how the new update had ruined the “story” function, my bones had turned to dust like the Nazi in Indiana Jones and the Last Crusade. I don’t necessarily feel a sense of “millennial solidarity”, except when millennials are being attacked by tabloids for being too sensitive (ie showing a level of compassion for people who don’t look like them).


As biodiversity is declining at an unprecedented rate, an important current scientific challenge is to understand and predict the consequences of biodiversity loss. Here, we develop a theory that predicts the temporal variability of community biomass from the properties of individual component species in monoculture. Our theory shows that biodiversity stabilises ecosystems through three main mechanisms: (1) asynchrony in species’ responses to environmental fluctuations, (2) reduced demographic stochasticity due to overyielding in species mixtures and (3) reduced observation error (including spatial and sampling variability). Parameterised with empirical data from four long-term grassland biodiversity experiments, our prediction explained 22–75\% of the observed variability, and captured much of the effect of species richness. Richness stabilised communities mainly by increasing community biomass and reducing the strength of demographic stochasticity. Our approach calls for a re-evaluation of the mechanisms explaining the effects of biodiversity on ecosystem stability.

\underline{Introduction}
\begin{itemize}
	\item Ecosystems undergo temporal stressors which impact their stability
	\item It seems intuitive that biodiversity increases stability with different species compensating for each other when lost. But there has been lots of debate about the relationship of diversity and stability since the 1970's
	\begin{itemize}
		\item This is mainly because while diversity increases stability of overall biomass it decreases stability of individual species abundances
	\end{itemize}
	\item A number of theories have been developed to explain diversities stabilising effect on aggregate ecosystem properties.
	\begin{enumerate}
		\item Statistical approach based on phenomenological relationships
		\item A stochastic approach describing population dynamics but not specifically species interactions
		\item A general population dynamical approach
		\item Specific models of interspecific competition 
	\end{enumerate}
	\item these all kinda describe whats going on but not a mechanism! $\Rightarrow$ This is still contentious
	\item These are not able to predict ecosystem stability from the properties of component species
	\item This study comes up with a new theory to do just that...
	\end{itemize}
\underline{Theoretical Model}
\begin{itemize}
	\item Discrete-time version of Lotka-Volterra model incorporating environmental and demographic stochasticity
	\item Description of model (not really important for me)
	\item Lots of maths and results...  
\end{itemize}
\underline{Discussion}
\begin{itemize}
	\item Their model explained 22-75\% variance in the aboveground community biomass in 4 long-term experiments
	\item Summed species covariances are unlikely to provide a mechanistic explanation for community stability
	\item Asynchrony of species environmental responses is the basic mechanism of the 'insurance hypothesis'
	\item Their model also shows these. Asynchronous species responses $\Rightarrow$ Greater community stability
	\item Reduced demographic stochasticity $\Rightarrow$ increased community biomass
	\begin{itemize}
		\item Species richness increases community biomass, though complimentary species or selection of more productive species $\Rightarrow$ known as overyielding
		\item Their study when tested on empirical data showed this was happening...
	\end{itemize}
	\item Effect of diversity on ecosystem stability through reduced observation errors
	\begin{itemize}
		\item If species biomasses are measured individually, the higher the diversity the more the error will even out across the whole community biomass...
		\item Common species weigh more on the variability than rare species.
		\item Maybe this is just a methodological problem?
		\item all this stuff is based on experiments in monocultures...
	\end{itemize}
\end{itemize}