\section*{New Approaches for delineating n-dimensional hypervolumes \citep{Blonder2018}}

\begin{enumerate}
	\item Hutchinson's n-dimensional hypervolume concept underlies many applications in contemporary ecology and evolutionary biology. Estimating hypervolumes from sampled data has been an ongoing challenge due to conceptual and computational issues.
	\item We present new algorithms for delineating the boundaries and probability density within n-dimensional hypervolumes. The methods produce smooth boundaries that can fit data either more loosely (Gaussian kernel density estimation) or more tightly (one-classification via support vector machine). Further, the algorithms can accept abundance-weighted data, and the resulting hypervolumes can be given a probabilistic interpretation and projected into geographic space.
	\item We demonstrate the properties of these methods on a large dataset that character- ises the functional traits and geographic distribution of thousands of plants. The methods are available in version $\geq$2.0.7 of the HYPERVOLUME R package.
	\item These new algorithms provide: (i) a more robust approach for delineating the shape and density of n-dimensional hypervolumes; (ii) more efficient performance on large and high-dimensional datasets; and (iii) improved measures of functional diversity and environmental niche breadth.
\end{enumerate}

\underline{Introduction}
\begin{itemize}
	\item Over the last decade there has been a number of studies using n-dimensional hypervolumes as a central concept
	\item Assuming a system can be characterised by a set of independent axes, these axes would constitute a n-dimensional euclidean space.
	\item Choosing how to delineate this n-dimensional space is a bit controversial. There has historically been differences when niche-data is used vs trait-data
	\item All depends on the  goals of the analysis
	\item Hypervolume R package uses a Monte Carlo approach. It can describe complex shapes, measure their volume and perform set operations	(to find distance, intersections etc).
	\item This paper develops the hypervolume concept and better algorithms to delineate them.		
\end{itemize}

\underline{New Hypervolume construction methods}
\begin{itemize}
	\item Two methods  
	\begin{itemize}
		\item Gaussian: this kernal decays towards zero in all directions, if you believe the data is an unbiased sample from a probability distribution then use this.
		\item Support-vector-machine: provides a smooth fit around the data. Use this if you think the extremes of the data represent the true bounds.
	\end{itemize}
	\item When calculating hypervolumes the max number of dimensions $n$ should be no more than $\log{m}$ where m is the number of data points.
	\item They use a hyperelliptical uniform sampling algorithm 
\end{itemize}

\underline{Other New Functionality}
\begin{itemize}
	\item hypervolume\_general\_model, this effectively maps an n-dimensional hypervolume to on-dimensional Euclidean space.
	\item Ability to weight data
	\item Ability to create geographic maps for species distribution modelling
\end{itemize}

\underline{Demonstration Analysis}
\begin{itemize}
	\item Functional trait analysis: Three trait values
	\item Niche Analysis: Three climatic variables
\end{itemize}