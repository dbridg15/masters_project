\section{Abstract}

Continued demand for resources has seen the conversion of natural landscapes to urban and agricultural uses increase. These land use changes, which are particularly prevalent  in the highly biodiverse tropical rainforests of Southeast Asia impact the structure of ecosystems and their resilience to environmental perturbation. The debate surrounding the relationship of biodiversity with stability remains contentious with both theoretical and empirical studies providing evidence on either side. Stability has traditionally been measured as the variability of communities through time, `community temporal stability', though a plethora of definitions exist. Advances in the delineation of \emph{n}-dimensional hypervolumes offers a new approach to studying ecosystem stability. Here I use species abundance data from the SAFE project in Sabah, Malasia to construct `community composition hypervolumes' for the taxonomic groups of trees, mammals and beetles at plots across a gradient of forest quality and logging history. I assess ecosystem stability using measures of hypervolume overlap between censuses, demonstrating a new approach to measuring ecosystem stability.
\\
\\

\textbf{Keywords:} 
\\ ecosystem stability, \emph{n}-dimensional hypervolumes, land-use change, SAFE.