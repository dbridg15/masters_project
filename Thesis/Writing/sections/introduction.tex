
\section{Introduction}

Ecosystems worldwide are experiencing escalating pressures from human influenced environmental changes including land-use intensification and climate change \citep{Hautier2015}. Conversion of natural landscapes to agriculture and urban environments has been driven by continued population growth and increased demand for resources \citep{Green2005, Foley2005} having significant impacts on global biodiversity \citep{Pimm1995}. These changes are shifting ecosystem communities away from historically stable states \hl{CITATION!} and impacting their resilience to future environmental perturbations \hl{CITATION!}. These losses in biodiversity lead to loss of ecosystem function and the ecosystem services on which we rely \citep{Diaz2006}. It is therefore vital for us to better understand how changes in land-use impact the stability of ecosystem communities and function, in order to better manage our landscapes and prevent the loss ecosystem services \hl{(CITATION)}. 


These changes are particularly prevalent in South East Asia – a global biodiversity hotspot \citep{DeBruyn2014} - where selective logging for timber and conversion to oil palm plantations has significantly changed the structure of the rainforests which support such a high diversity \hl{(CITATION)}. This logging has left fragments of forest which support only a proportion of the species which would be present in continuous forest \hl{(CITATION)} and palm-oil plantations – which support a much lower species richness than the forests which they replace. While these practices have been essential for the economic development of the region \hl{(CITATION)} – them impact is significant (AGB and Biodiversity) - impacting things like carbon sequestration, hydrology etc. \hl{(CITATION)} also a net release of CO2. With the growing demand for palm-oil it is unlikely that conversion of forest will cease in the near future and so there is a need to better manage \hl{(CITATION)}. 


Stability can have a multitude of meanings when referred to in the context of ecosystems as \cite{Pimm1984} laid out. It can be measured as: resilience; the time for the ecosystem to return to its equilibrium after a perturbation, resistance, the amount the ecosystem in changed by a perturbation, persistence; the time it takes before the ecosystem responds to the perturbation and variance; the degree to which the ecosystem varies though time. On top of this stability can be measured at a range of levels of complexity (species richness, community structure) which further multiply the possible meaning of ‘stability’ \cite{Pimm1984,Lehman2000}.  It is therefore essential for studies to be explicit in what they are measuring and at which level of complexity as it is likely any relationships will differ between them. Temporal stability \citep{Tilman1999, Lehman2000} is a measure of stability related to variance, it is defined as the mean abundance (\mu) of the system divided by the standard deviation (\sigma) of the variation in abundance through time \hl{(equation 1)}, such that ecosystems with very high temporal variation will have a stability close to zero and where there is no variation stability will approach infinity. 

The relationship between ecosystem stability and biodiversity has been argued over in ecology for decades. Initial ??thoughts?? from \cite{Elton1958} and \cite{MacArthur1955} suggest that increasing biodiversity and complexity within the system would directly lead to more stability, This seems intuitive as with greater species richness there is redundancy for species to fill function when others are lost. However further theoretical work contradicted these ideas \hl{(CITATION!?)}.  