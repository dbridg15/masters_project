
\section{Introduction}

Ecosystems worldwide are experiencing escalating pressures from human influenced environmental changes including land-use and climate change \citep{Hautier2015}. Conversion of natural landscapes to agriculture and urban environments has been driven by continued population growth and increased demand for resources \citep{Green2005, Foley2005} significantly impacting global biodiversity \citep{Pimm1995}. These changes are shifting ecosystem communities away from historically stable states \citep{Hautier2015} and reducing their resilience to future environmental perturbations \citep{Oliver2015}. These biodiversity losses lead to reduced ecosystem function and services on which we rely \citep{Diaz2006, MillenniumEcosystemAssessment2005}.The impacts of these environmental changes are hard to predict \citep{Carpenter2009}, though globally ecosystem services are deteriorating \citep{Mace2012}.

Land-use changes are particularly prevalent in the tropical forests of Southeast Asia, a global biodiversity hotspot \citep{DeBruyn2014},  where selective logging for timber and conversion to oil palm plantations has significantly changed the structure of the rainforests which support such high biodiversity \citep{Gibson2011}. This logging has left degraded forest, fragments of old growth and oil palm plantations, all of which support a much lower species diversity than the continuous forests they replace \citep{Fitzherbert2008, Haddad2015} . These practices have been essential for the economic development of the region \citep{Basiron2007} but their impact has been significant on ecosystem processes \citep{Koh2011, Schleuning2011, Ewers2015}.  With demand for palm oil continuing to grow it is essential we learn to better manage these landscapes to mitigate any further impact \citep{Turner2008}.

\hl{MAKE BETTER} Large scale and long-term ecological field experiments are needed to provide this understanding, the Stability of Altered Forest Ecosystems Project (SAFE) is such a project \citep{Ewers2011}. Located in the Sabah, Malaysia the experiment covers a gradient of levels of forest modification with a fractal sampling design covering forest fragments of different sizes. Covering forest which is in the process of conversion to oil palm plantations it provides the perfect opportunity to study the impacts of this land-use change.

\subsection{Ecosystem stability}

Ecosystem stability can have a multitude of meanings \citep{Pimm1984}. It can be measured as: resilience; the time for the ecosystem to return to its equilibrium after a perturbation; resistance, the amount an ecosystem in changed by a perturbation; persistence, the time it takes before an ecosystem responds to the perturbation and variance, the degree to which an ecosystem varies though time. It is also possible to think of stability in terms of the number of possible stable states of a community, with more potential stable states resulting in a less stable system, increasing the likelihood of sudden shifts between ecosystem states \citep{Scheffer2001}. On top of this stability can be measured at a range of levels of complexity (e.g. species richness or community structure) which further multiplies the possible meaning of ‘stability’ \citep{Pimm1984,Lehman2000}.  It is therefore essential for studies to be explicit in what they are measuring and at which level of complexity as it is likely any relationships will differ between them. Temporal stability \citep{Tilman1999} is a measure of stability based on variance, it is defined as the mean abundance (\mu) of the species divided by the standard deviation (\sigma) of the variation in abundance through time $ \text{S = \mu/\sigma} $. This can be modified to assess the stability of a whole community (equation \ref{equ:temporal_stability}) \citep{Lehman2000}. This equation takes into account the variance of the system as well as how each possible pair of species covary together. From this we can see that the stability of a community increases with higher abundance and reduced variance and covariance. This equation can also be used to look at stability through space in place of time.

\vspace{0.3cm}

\begin{equation} \label{equ:temporal_stability}
\text{S\textsubscript{T}} = \frac{\text{\mu\textsubscript{T}}}{\text{\sigma\textsubscript{T}}} = \frac{\text{\Sigma \, Abundance}}{\sqrt{\text{\Sigma \, Variance + \Sigma \, Covariance}}}
\end{equation}


\vspace{0.3cm}

The relationship between ecosystem stability and biodiversity has been argued over in ecology for decades. Initial thoughts from \cite{Elton1958} and \cite{MacArthur1955} suggested that increasing biodiversity and complexity within a system would directly lead to higher stability. This seems intuitive as with greater species richness there is redundancy for species to fill function when others are lost, known as the insurance hypothesis \citep{Yachi1999}. However further theoretical work contradicted these ideas \citep{May1973} finding that increased diversity and community interactions decreased the stability of a system. Empirical studies have found that there is a positive relationship between diversity and temporal stability \citep{Tilman1994, Tilman2006}, with a systematic review by \cite{Ives2007} showing that a positive relationship was found in 69\% of studies on the topic. A number of explanations as to why diversity may increase stability have been proposed \citep{McCann2000}. \cite{Doak1998} suggested the Averaging effect, showing that even excluding interactions increasing species numbers on average reduced the variation in biomass due to statistical averaging, while \cite{Tilman1998} countered this with the Negative-Covariance effect which suggests that where species have a negative covariance, perhaps due to interspecific competition, this would generally stabilise the community. Regardless, it is clear that there is no simple link between stability and diversity \citep{Goodman1975}.

\hl{PARAGRAPH ON WHY TAXA!?}

\subsection{The \emph{n}-dimensional hypervolume}

An \emph{n}-dimensional hypervolume defines a volume in higher dimensional space. They have been used throughout ecology to represent systems with many components, allowing for a holistic approaching to assessing these components in combination.  \cite{Hutchinson1957} introduced the concept, using them to evaluate species niches. In his description each axis represents some independent environmental variable, physical or biological, with the hypervolume delineating the euclidean space that represents the environmental conditions which would allow the species to perpetuate. Since then the concept has been extended to incorporate axis of different kinds (climate, morphological traits, functional traits, etc.) measured at varying levels (communities, populations, species) to delineate hypervolumes in \emph{n}-dimensional space. Allowing questions about community composition \citep{Barros2016}, niche shifts \citep{Tingley2014, Blonder2015, Jackson2000, Evans2009}, functional richness \citep{Lamanna2014} and clade diversification \citep{Sidlauskas2008} to be explored. Recent developments allowing hypervolumes to be better delineated, especially in higher dimensions \citep{Blonder2014, Blonder2017b, Junker2016} provide potential for hypervolumes to become a common tool within ecology.

 
\cite{Barros2016} demonstrated that it is possible to use this concept of \emph{n}-dimensional to study the stability of ecosystems. They suggest that by describing an ecosystem state using axis from a range of ecosystem components before, during and after perturbation, the impact of the perturbation on the ecosystem state can be assessed. Using measures of hypervolume overlap, volume change and centroid movement the magnitude of change can be quantified. They highlight that studies on the rate of return to pre-perturbation state or new stable state could also be investigated. As a proof of concept they simulated climate change on European alpine plant communities and constructed two sets of hypervolumes at each time-step, one `community hypervolume' constructed using abundanced of plant functional groups, after a PCA to reduce dimensionality and the other using specific traits as the axis. Using measures of hypervolume change they assessed how the simulated ecosystem responded to the perturbation of climate change.


\subsection{Aims}

In this study I will use \emph{n}-dimensional hypervolumes, constructed to represent the community composition of trees, mammals and beetles at forest plots of different quality and logging history from SAFE. Comparing hypervolumes with a measure of overlap I will test the hypothesis that:

\begin{itemize}
	\item ecosystems with greater biomass will be more stable \citep{Tilman2006}, with the expectation that plots with higher biomass will have higher levels of hypervolume overlap.
	\item longer lived taxa will have more stable communities with greater hypervolume overlap
\end{itemize}
 
 I will investigate how measures of stability using hypervolumes compare to the more established community temporal stability \citep{Lehman2000}.
 	