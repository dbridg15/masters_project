
\section{Materials \& Methods}


\subsection{Data Sources \& Preparation}

	\textbf{Trees:}
	\begin{itemize}
		\item 5x5 grids
		\item 9 plots across gradient of logging
		\item AGB calculated as data
		\item plots = hypervolumes
		\item subplots = points within hypervolume
		\item how was th data collected? \citep{Riutta2018}
	\end{itemize}
	
	\textbf{Mammals:}
	\begin{itemize}
		\item traps act as points
		\item totals for years (how many years?)
		\item which plots?
		\item AGB is closest fractal2 point by distance (where does this data come from?)
		\item how was th data collected? \citep{Wearn2017}
	\end{itemize}
	
	\textbf{Beetles:}
	\begin{itemize}
		\item 20 plots???? seems wrong might need to amalgamate...
		\item traps act as points
		\item 3 discrete censuses (when were they?)
		\item AGB = corresponding fractal2 point (which would be closest by distance anyway) - median...
		\item how was the data collected? ????
	\end{itemize}
		
\subsection{Dimension Reduction of Community Composition}
	Need to say why I actually did that!
	Did it on raw abundances!
	I did a Principal Component Analysis using the 'rda' function from the R package vegan \citep{Oksanen2018}.
	Separate one for each taxa but all plots/subplots/traps and censuses were done together so they would be comparable.
	Chose the top 3 principal components - somehow need to justify!

	could also talk about ZIFA?
	
\subsection{Construction of Hypervolumes}
	Used the R package 'hypervolumes' \citep{Blonder2017a}
	all hypervolumes were constructed in three dimensions.
	Briefly explain the way they are constructed! (see \cite{Blonder2014, Blonder2017b} for more details)

\subsection{Comparison of Hypervolumes}
The Jaccard similarity (Overlap) of hypervolumes was used - each census step... (census steps effectively acting repeats).
Also centroid change...

\subsection{Comparing Taxa}
I did it...

\subsection{Effect of AGB}
I did it...

\subsection{Spatial Stability}
Was calculated for each plot/census - same Jazz...