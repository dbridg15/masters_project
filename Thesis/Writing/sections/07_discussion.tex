
\section{Discussion}

Para1 - Brief description of results.
\begin{itemize}
	\item hypervolume overlap was significantly higher in trees than either mammals or beetles, same for temporal community stability - so trees are more stable
	\item why might that be? \hl{CITATION!}
	\item No significance was found when looking at the effect of AGB on the measures of stability, no significant slopes and no interaction with taxa. With the exception of temporal stability and trees...
	\item why might this be? \hl{CITATION!}
\end{itemize}

Para2 - yay I found a significant result - trees are more stable than beetles or mammals (who would've thought!)
\begin{itemize}
	\item I know right crazy shit!
	\item \hl{find that paper}
	\item less turnover of trees then of mammals and beetles - all to do with generation time
	\item there was that paper about turnover increasing! \citep{Phillips1994} and another somewhere!\
	\item any literature on stability of mammals and beetles - specifically in rainforest - must be something on diversity at least
	\item maybe cause beetles are at family level they appear more stable?
	\item have very few and disparate mammal plots so cant really say much about them!
\end{itemize}

Para3 - no significance on AGB! - Why might that be?
\begin{itemize}
	\item Given \citep{Tilman2006} and others we would have expected a positive relationship with stability and AGB - but alas it is not so, even if some of the figures make it look like something is going on!
	\item explain the thoughts behind the hypothesis
	\item why might my study not have found anything! (Apart from trees and temporal stability - but really thats a big dodge)
	\item PCA only explained some of the variation (but it explained most of beetles! and enough of mammals).
	\item AGB measures arnt perfect.. act as a continuous scale of forest degredation but might not actually have anything to do with the quality of the forest - and they are very much over-simplified in this study!
\end{itemize}


Para4 - in general hypervolumes performed pretty badly in these circumstances - potential reasons (compare with other measures of stability...)
\begin{itemize}
	\item correlated very strongly overall with temporal stability - more to do with the fact that the groups followed the same pattern. Again the only group significant on its own was trees!
	\item temporal stability is all about number and abundance of species - which is effectively what my hypervolumes are about so should give similar results right?
	\item other things about temporal stability...
\end{itemize}

Para5 - more on the general limitations of the study
\begin{itemize}
	\item these arent the hypervolumes you're looking for.
	\item Only one kind of hypervolume
	\item lots of gaps in the data (missing censues, empty traps etc, makes it hard to interpret)
	\item constrined to 3-dimensions in this case
	\item only 3 groups!
	\item only family data for beetles
	\item didnt use an algorithm to get the bandwidth... (very bad of me!)
\end{itemize}

Para6 - don't give up hope there is still plenty of scope for hypervolumes in the future! - Just imagine if I had had functional data!
\begin{itemize}
	\item lets face it hypervolumes were not designed for what i used them for!
	\item meant to have a whole load of different componants together - which would have provided much more interesting results!
	\item talk about other studies on hypevolumes - links to stability
	\item and other studies on stability - how could they be improved by using hypervolumes?
\end{itemize}


Conclusions...
\begin{itemize}
	\item cant say anything particularly concrete from this study!
	\item still lots of potential for the future though!
\end{itemize}