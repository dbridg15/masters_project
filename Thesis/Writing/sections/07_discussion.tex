
\section{Discussion}

This project attempted to study ecosystem stability in a new way, using n-dimensional hypervolumes to represent the state of ecosystem communities and assess the variation in these hypervolumes between censuses at plots of varying forest quality.  

Hypervolume overlap between censuses was found to be significantly higher in tree communities than in either mammals or beetles, the other two taxa studied. This suggests that trees are overall more stable as a taxa. However, no significant relationship was found between the AGB (a continuous measure of forest quality) of the plot and the hypervolume overlap and so stability of communities within. This is contrary to what is predicted by theory \citep{MacArthur1955, Hughes2000a} and has previously been found by empirical studies \citep{Tilman2006, Ives2007}, where increased biomass has been associated with greater stability of ecosystems. These measures of stability using hypervolume overlap were also compared with the more traditional measure of community temporal stability \citep{Lehman2000} which produced similar results. Using this more traditional measure still found no relationship between AGB and stability. 

Trees were found to be significantly more stable, with greater levels of hypervolume overlap than either of the other taxa tested. Given that stability was measured at the scale of years this is not a surprising result as the generation length is higher and so turnover of trees is much lower than either of the other groups \citep{Connell1983, Pimm1984}. Following the same logic you would expect that mammals would show a greater level of stability than beetles, however this result was not found. This may be due to the fact that the beetle data was only identified to family level and so much of the turnover and changed in species composition may have been hidden. Identifying beetles to species level and rerunning the analysis may lead to reduced level of hypervolume overlap and stability of the group.

It is encouraging that hypervolume overlap produced similar results and correlated strongly with the community temporal stability measures, suggesting that \emph{n}-dimensional hypervolumes do provide potential for future studies in ecosystem stability. However the real power of n-dimensional hypervolumes lies in the fact that a whole host of components can be used as axis to define their shape \citep{Blonder2014, Barros2016}. This study did not take full advantage of this fact, only using axis derived from species abundances to delineate the hypervolumes. Looking at more and varied components provides the potential to study and understand stability more complexly. For example the functional stability of an ecosystem is likely much more important to ecosystem services than the specific species which make up the community.

This project faced a number of issues. Due to data constraints the community composition hypervolumes had a maximum of 3 dimensions, which in effect means all hypervolumes in this study are misnomers and should in fact be referred to as `community composition volumes'. The study attempts to compare hypervolumes of different taxa, constructed from very different data, hypervolume overlap (a proportion) was used instead of any direct measure of distance or volume as these would not be comparable between hypervolumes constructed from the output of different PCAs. The PCAs also failed to explain all the variation in community composition within their first 3 components, most notably for trees where only 38.98\%of variation was explained by the components which were used to delineate the hypervolumes. The study was also limited by the taxa for which appropriate data was available, in the future extending a similar analysis to a wider variety of taxa as well as ecosystem components could provide more detailed insights into how ecosystems respond to environmental perturbations.

The study focused on specific definition of stability, variation though time, in this case variation in the composition of the community through time. Hypervolumes have the potential to be used to asses other stability measure, such as resilience or persistence, as well as investigating the movement of ecosystems to different `stable states' \citep{Barros2016}. \emph{N}-dimensional hypervolumes clearly represent a new approach to assessing the stability of ecosystems. Not without their own set of issues they provide the ability to investigate the impact of perturbations on a variety of ecosystem components at the same time.

