\section*{Oil Palm Research in Context: Identifying the Need for Biodiversity Assessment \cite{Fitzherbert2008}}

\subsection*{Abstract}
Oil palm is one of the world’s most rapidly increasing crops. We assess its contribution to tropical deforestation and review its biodiversity value. Oil palm has replaced large areas of forest in Southeast Asia, but land-cover change statistics alone do not allow an assessment of where it has driven forest clearance and where it has simply followed it. Oil palm plantations support much fewer species than do forests and often also fewer than other tree crops. Further negative impacts include habitat fragmentation and pollution, including greenhouse gas emissions. With rising demand for vegetable oils and biofuels, and strong overlap between areas suitable for oil palm and those of most importance for biodiversity, substantial biodiversity losses will only be averted if future oil palm expansion.

\subsection*{Oil palm: one of the world's most rapidly expanding crops}
\begin{itemize}
	\item agricultural intensification is a significant threat to biodiversity and vegetable oils are some of the fastest growing agricultural commodities
	\item Malaysia and Indonesia contribute 80\% of global production
	\item the impact of palm oil on ecology is directly related to the amount of deforestation which it likely causes deforestation (debated amoung scientists and industry)
	\item and the biodiversity value of tropical forest vs that of palm oil plantations
\end{itemize}


\subsection*{Contribution of oil palm expansion to deforestation}
\begin{itemize}
	\item it is very difficult to quantify with poor data and complex causes of land-cover change
	\item four ways in which oil palm expansion could contribute to deforestation:
	\begin{enumerate}
		\item primary motive for clearance
		\item replacing previously degraded forest
		\item as a combined economic enterprise with timber etc.
		\item indirectly through generating improved road access to previously inaccessible forest 
	\end{enumerate}
	\item it is therefore pretty easy to misjudge how the oil-palm came to be there, it may not have directly replaced pristine forest
	\item \textbf{Malaysia:} first commercial plantation in 1917 - expanded to Sabah and associated with logging - Expansion of 1.8 million ha between 1990-2005 (while 1.1m ha were lost)
	\item \textbf{Indonesia:} commercial plantations from 1911 but expansion beyond Sumatra not till the 1980s. Some crap loopholes mean that timber/pulp companies can log forest under the pretence that they will plant but then never do... And illegal plantations pop up in protected areas. Conversion to oil plantation could account for 16\% of deforestation between 1990 and 2005 - but this estimate has very high uncertainty and could very well be an under or over-estimate
	\item while its historical contribution to deforestation is uncertain, its future potential is significant!
\end{itemize}


\subsection*{Effects of converting forests to oil palm plantations}
\begin{itemize}
	\item two possibilities:
	\begin{enumerate}
		\item palm oil cannot sustain the same biodiversity as forest and so the priority should be to reduce deforestation
		\item plantations can be managed so they can sustain the same forest species while maintaining high yields. In which case we should focus efforts into enhancing this biodiversity at plantations
	\end{enumerate}
	\item they did a literature review to assess this...
	\item palm plantations are structurally less complex than natural forest with uniform tree ages, lower canopy, sparse undergrowth, a less stable microclimate and greater human disturbance
	\item palm is also cleared and replanted on a 25-30 year rotation
	\item \textbf{Species Richness:} Palm had consistently fewer than half as many vertebrate species as forest while invertebrates were more variable
	\item across all taxa only 15\% of species in primary forest were also found in oil palm plantations
	\item \textbf{composition:} large differences in faunal species between palm and forest. Not a random subset, specialised diets, reliant in forest habitat features (i.e. canopy), small range sizes, high conservation concern.
	\item plenty of caveats as to why these are likely conservative estimates
	\begin{itemize}
		\item harder to detect species in forest (complex and canopy)
		\item transient species in forest edges coming into plantation
		\item lag between habitat loss and extinction
	\end{itemize}
\end{itemize}


\subsection*{Comparison with other land uses}
\begin{itemize}
	\item oil palm is a poor substitute for forest - even degraded forest and has worse species richness than most other agricultural alternatives (i.e. rubber)
\end{itemize}


\subsection*{Landscape scale effects}
\begin{itemize}
	\item palm plantations can act a a barrier to movement/migration
	\item forest fragments in oil plantation supported fewer than half the ant species as nearby continuous forest
	\item fragments also have more 'tramp' species (invasive species)
	\item fragmentation also massively increases the number of forest edges
\end{itemize}


\subsection*{What can be done to mitigate the impacts?}
\begin{itemize}
	\item we need good policy!
	\item strategies for land allocation - put plantations in already shit land
	\item planters need to now where they will cause the least ecological damage!
\end{itemize}