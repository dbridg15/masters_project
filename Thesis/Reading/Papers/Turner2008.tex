\section*{How will oil palm expansion affect biodiversity? \cite{Fitzherbert2008}}

\subsection*{Abstract}
Oil palm cultivation is frequently cited as a major threat to tropical biodiversity as it is centered on some of the world's most biodiverse regions. In this report, Web of Science was used to find papers on oil palm published since 1970, which were assigned to different subject categories to visualize their research focus. Recent years have seen a broadening in the scope of research, with a slight growth in publications on the environment and a dramatic increase in those on biofuel. Despite this, less than 1\% of publications are related to biodiversity and species conservation. In the context of global vegetable oil markets, palm oil and soyabean account for over 60\% of production but are the subject of less than 10\% of research. Much more work must be done to establish the impacts of habitat conversion to oil palm plantation on biodiversity. Results from such studies are crucial for informing conservation strategies and ensuring sustainable management of plantations.

\subsection*{Introduction}
\begin{itemize}
	\item palm oil has grown to be ubiquitous (need synonym!) in the food and chemical industry (soap etc.) 
	\item this has led to a significant increase in the space that is used to grow it!
	\item as a tropical crop its production is generally centred in highly biodiverse localities (Malaysia and Indonesia are the biggest producers)  with high levels of endimism - \citep{Basiron2007}
	\item new uses of palm oil being investigated - may lead to even more growth in production
	\item they want to look at how much research is actually being done on the impacts of palm cultivation
\end{itemize}

\subsection*{Methods}
\begin{itemize}
	\item they used Web of Science looking for ````palm oil" or ``oil palm""
	\item 3056 publications between 1970 and 2006, were categorised based on title, abstract and key words
	\item compared with ``agriculture" and ````vegetable crop name" \& ``oil""
\end{itemize}


\subsection*{Results and Discussion}
\begin{itemize}
	\item palm oil oil and soybean contribute 60\% of global production of vegetable oil but only 10\% of the research interest
	\item very low number of the publications look at biodiversity (0.75\%) or even environmental issues of palm production (2.06\%)
	\item those that do look at biodiversity look at large mammals and birds
	\item management to enhance biodiversity is not neccessarily contrary to increased oil production (it might reduce soil erosion and flooding for example)
\end{itemize}