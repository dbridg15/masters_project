\section*{Global Consequences of Land Use \citep{Foley2005}}

\subsection*{Abtract}
Land use has generally been considered a local environmental issue, but it is becoming a force of global importance. Worldwide changes to forests, farmlands, waterways, and air are being driven by the need to provide food, fiber, water, and shelter to more than six billion people. Global croplands, pastures, plantations, and urban areas have expanded in recent decades, accompanied by large increases in energy, water, and fertilizer consumption, along with considerable losses of biodiversity. Such changes in land use have enabled humans to appropriate an increasing share of the planet’s resources, but they also potentially undermine the capacity of ecosystems to sustain food production, maintain freshwater and forest resources, regulate climate and air quality, and ameliorate infectious diseases. We face the challenge of managing trade-offs between immediate human needs and maintaining the capacity of the biosphere to provide goods and services in the long term.
60s
\subsection*{The Rest}
\begin{itemize}
	\item while land-use practices vary across the world, their consequences are the same - acquisition of resources for immediate human needs at the expense of degrading environmental conditions.
	\item affects everything
	\begin{itemize}
		\item climate
		\item carbon cycle
		\item hydrology
		\item nutrients in biosphere
		\item biodiversity
		\item etc.
	\end{itemize}
	\item its a catch 22, we need the land use to provide us with stuff we need/want but we need ecosystems to survive too!
	\textbf{Food}
	\begin{itemize}
		\item Croplands and Pasture cover 40\% of the worlds land surface, rivalling forest cover
		\item grain production has doubled since the 1960s
		\item fertilize use has skyrocketed! - Very band for water quality
		\item soil erosion, decreased fertility, overgrazing
		\item in short: Short-term increases in food production are leading to long-term losses in ecosystem services
	\end{itemize}
	\textbf{Freshwater Resources} (NA)
	
	\textbf{Forest Resources}
	\begin{itemize}
		\item net loss of *alot* of forest in the last few centuries
		\item managed lands replacing *natural* forest 
		\item things such as forest grazing and road expansion can have a significant impact without changing forest area.
		\item introduction of pests
		\item reforestation is also a thing!
		\item the biomass of European forests has increased by ~ 40\% while the area has remained similar.
	\end{itemize}

	\textbf{Confronting the Effects of Land Use}
	\begin{itemize}
		\item we need good policy!
	\end{itemize}

\end{itemize}

