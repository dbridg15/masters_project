\section*{Forest quality, forest area and the importance of beta-diversity for protecting Borneo's beetle biodiversity \citep{Sharp}}

\begin{enumerate}
	\item The lowland forest of Borneo is threatened by rapid logging for timber export and clearing for the expansion of timber and oil palm plantations. The combination of these two processes leaves behind landscapes dotted with small, often heavily degraded forest fragments. The biodiversity value of such fragments, which are easily dismissed as worthless, is uncertain.
	\item We collected 187 taxa of staphylinid beetles across a land-use gradient in Sabah, Malaysia, spanning pristine tropical lowland forest to heavily-degraded forest. Using these data, we identified shifts in alpha-, beta-, and gamma-diversity in response to forest quality and forest area, and applied our findings from continuous expanses of forest to make predictions on hypothetical forest fragments.
	\item We found that maintaining high forest quality is important for conserving rare taxa (those important for conserving biodiversity per se), and that very small fragments (10 ha) are likely to harbour the same richness of staphylinids as larger fragments (100 ha) of the same forest quality. We estimate a decline in staphylinid richness of just 16\% following the removal of 90\% of the vegetation biomass from a fragment within this area range.
	\item Maintaining large forest areas is important for conserving common taxa, likely to be more important for conserving ecosystem functioning. Our analyses suggest that 100 ha fragments of heavily-degraded forest can support the same or greater diversity of common taxa as 100 ha fragments of pristine forest. We find that reducing 100 ha fragments to 10 ha fragments will likely result in the loss of just 11\% of common taxa diversity.
	\item Despite significant declines in alpha-diversity, beta-diversity within small rainforest fragments will likely partially mitigate the loss of gamma-diversity, reinforcing the concept that beta-diversity is a dominant force determining the conservation of species in fragmented landscapes.
	\item Synthesis and applications. In contrast to previous findings on larger animals, our results suggest that even small fragments of degraded forest might be important reservoirs of invertebrate biodiversity in tropical agriculture landscapes. These fragments should be conserved where they occur and form an integral part of management for more sustainable agriculture in tropical landscapes.
\end{enumerate}


\underline{Introduction}
\begin{itemize}
	\item Borneo is super diverse, but this diversity is under threat.
	\item Need to understand the consequences of land use change to make `informed management recommendations'
	\item Previous studies show that remaining fragments of forest support a greater diversity than palm plantation (though far less than pristine continuous forest)
	\item Important that at least fragments are spared.
	\item Logged forest is also important! Now treated as a discrete habitat type (Old Growth, Logged, Palm Plantation)
	\begin{itemize}
		\item But! logged forest is very variable in disturbance type/amount. It is better to put habitat quality on a continuous scale to better understand response to disturbance.
	\end{itemize}
	\item measuring impact of habitat change is difficult cause it changes between different measures of diversity.
	\item Diversity:
	\begin{itemize}
		\item at one point = Alpha
		\item between points = Beta
		\item Total = Gamma
	\end{itemize}
	\item studies usually look at species richness but this disregards the numbers of the different species...
	\item You want to protect rare species but also common species which are probably more important for ecosystem functioning.
	\item beta diversity is often neglected (and has issues) given this we remain pretty uninformed on the response of beta-diversity to land use change in Borneo but is really important especially when looking at spatial differences in diversity
	\item there are equations which allow the comparison of different measures of diversity
	\item its good to study a taxon which is abundance and ecologically important (enter beetles)
	\begin{itemize}
		\item very specious abd functionally diverse
	\end{itemize}
	\item uses Jost's diversity index to quantify impacts of forest quality and forest area on the richness and community composition.
\end{itemize}

\underline{Material \& Methods}
\begin{itemize}
	\item Study Site
	\begin{itemize}
		\item SAFE in Sabah Malaysia
		\item fractal sampling pattern specifically designed to study diversity at multiple spatial scales
		\item logged forest with widely varying quality
	\end{itemize}
	\item Insect Sampling
	\begin{itemize}
		\item traps were combo (pitfall, flight-interception and malaise)
		\item sampled twice
		\item Staphylinid beetles identified to lowest taxonomic level. Aleocharinae removed from analysis as too difficult to identify
	\end{itemize}
	\item Calculating Diversity metrics
	\begin{itemize}
		\item used josts equations
		\item removed traps combinations with no staphylinid beetles
		\item combinations of three traps were used as the simplest method of generating 2D shapes
		\item three weightings (equal, favour rare, favour common)
		\item these three weightings encompass arguments around species richness/diversity stuff...
	\end{itemize}
	\item Defining forest area and forest quality variables
	\begin{itemize}
		\item used AGB - depending on where the three points fell in the fractal design depended on how AGB was calculated (mean of different sites)
		\item each three-trap combination was assigned a continuous forest quality value.
		\item did some trickery to make sure forest area and AGB were nor correlated.
	\end{itemize}
	\item selecting models of forest area and forest quality
	\begin{itemize}
		\item Generalised linear mixed models were fitted to predict abundance from forest quality.
		\item model selection using BIC
	\end{itemize}
	\item creating nominal forest fragments
	\begin{itemize}
		\item ...
	\end{itemize}
\end{itemize}

\underline{Results}
\begin{itemize}
	\item ...
\end{itemize}

\underline{Discussion}
\begin{itemize}
	\item protecting primary forest is best for conserving species abundance and richness
	\item preserving forest area is better to maximise the diversity of common species which are likely to contribute more to ecosystem functioning
	\item habitat quality had a greater impact than area on invertebrate taxa richness.
	\item more area = more specialist species!
	\item forest areas < 116ha, habitat quality is more important than size  
	\item beta diversity has increased in some poor forest quality plots not because its more spatially heterogeneous but cause common species are less abundant.
	\item beta diversity is important for overall species richness and gamma diversity
	\item in highly disturbed forest alpha diversity is lower suggesting that the community is dominated by a small number of taxa
	\item evidence that logging has changed the makeup of the community but no overall reduction in diversity.
	\item no effect of habitat shape found
\end{itemize}