\section*{Vertical stratification of adult mosquitoes (Diptera: Culicidae) within a tropical rainforest in Sabah, Malaysia \citep{Brant2016} - \textit{}}

\subsection*{Abstract}
\textbf{Background:} Malaria cases caused by \textit{Plasmodium knowlesi}, a simian parasite naturally found in long-tailed and pig-tailed macaques, are increasing rapidly in Sabah, Malaysia. One hypothesis is that this increase is associated with changes in land use. A study was carried out to identify the anopheline vectors present in different forest types and to observe the human landing behaviour of mosquitoes. 

\textbf{Methods:} Mosquito collections were carried out using human landing catches at ground and canopy levels in the Tawau Division of Sabah. Collections were conducted along an anthropogenic disturbance gradient (primary forest, lightly logged virgin jungle reserve and salvage logged forest) between 18:00 and 22:00h.

\textbf{Results:} \textit{Anopheles balabacensis}, a vector of \textit{P. knowlesi}, was the predominant species in all collection areas, accounting for 70\% of the total catch, with a peak landing time of 18:30–20:00h. \textit{Anopheles balabacensis} had a preference for landing on humans at ground level compared to the canopy (p < 0.0001). A greater abundance of mosquitoes were landing in the logged forest compared to the primary forest (p < 0.0001). There was no difference between mosquito abundance in the logged forest and lightly logged forest (p = 0.554). A higher evening temperature (p < 0.0001) and rainfall (p < 0.0001) significantly decreased mosquito abundance during collection nights.

\textbf{Conclusions:} This study demonstrates the potential ability of \textit{An. balabacensis} to transmit \textit{P. knowlesi} between canopy-dwelling simian hosts and ground-dwelling humans, and that forest disturbance increases the abundance of this disease vector. These results, in combination with regional patterns of land use change, may partly explain the rapid rise in \textit{P. knowlesi} cases in Sabah. This study provides essential data on anthropophily for the principal vector of \textit{P. knowlesi} which is important for the planning of vector control strategies.


\subsection*{Introduction}
\begin{itemize}
	\item It has been proposed that land-use change has increased the interaction of humans with malaria vectors due to the encroachment of humans into previously forested areas.
	\item As the transmission of malaria is increasing its important to identify the vectors and their biting preference.
\end{itemize}


\subsection*{Methods}
\begin{itemize}
	\item Study Site
	\begin{itemize}
		\item Tawau division of Sabah, Malaysia
		\item Three areas along forest disturbance gradient; primary lowland, virgin jungle reserve, twice logged disturbed
		\item Three survey points with a minimum of 500m separation were used at each site.
		\item Survey points are a subset of SAFE 
	\end{itemize}

	\item Data Collection
	\begin{itemize}
		\item April to July 2014
		\item Human landing catches between 18:00 and 22:00
		\item At ground and (2/3) canopy height
		\item All mosquitoes which landed on the surveyor were collected and identified
	\end{itemize}

	\item Meteorological data
	\begin{itemize}
		\item Air temp and relative humidity measured at each site
		\item nightly rainfall from nearest rain gauges
		\item lunar illumination, cloud cover, 'unusual climatic events' also recorded
	\end{itemize}

	\item Data Analysis
	\begin{itemize}
		\item Calculated measures of diversity (simpsons and shannons)
		\item And species accumulation curves
		\item measures of 'true richness' were also predicted
		\item Generalised Linear Mixed Effects model to analyse the impact of canopy heigh and forest quality on mosquito abundance.
		\item Compared abundance of vector and non-vector species
		\item 
	\end{itemize}

\end{itemize}

\subsection*{Results}
\begin{itemize}
	\item Mosquito Abundance
	\begin{itemize}
		\item 807 mosquitoes collected
		\item 39 nights
		\item measures of diversity differed between forest types - more species in logged forest and at ground level
	\end{itemize}
\end{itemize}

\textbf{Stopped being relevant}
