\section*{The Stability of Altered Forest Ecosystems Project: Investigating the design of Human-Modified Landscapes for Productivity and Conservation \citep{Turner2012}}

\subsection*{Abstract}
With growing global demands for palm oil, there is mounting pressure on limited natural resources to support the dual services of agricultural productivity and maintenance of biodiversity. Balancing these two services requires detailed research on four themes: i) the impacts on biodiversity of forest conversion and fragmentation, ii) which factors drive these changes in biological communities, iii) what impacts changes have on ecosystem functioning and, iv) the management and design of multifunctional landscapes. Such questions are often difficult to answer as data must be collected at the landscape scale and over long time periods. The Stability of Altered Forest Ecosystems (SAFE) Project (see www.safeproject.net for more details) is a ground-breaking scientific study based in Sabah, Malaysia which investigates the impacts of forest conversion to oil palm on biodiversity, ecosystem functioning and productivity. Funded by the Sime Darby Foundation with support from Benta Wawasan and the Sabah Forestry Department, the project is a collaboration between research institutions and the oil palm industry. The project takes advantage of a 7 900 ha area of forest which was scheduled for conversion to oil palm in 2012, allowing the consequences of habitat conversion to be directly measured. Now at the beginning of its third year, the SAFE Project is already yielding results of direct relevance to tropical conservation and plantation management. As well as a core team of nearly thirty researchers and research assistants working full-time on the project, SAFE has also provided a platform for collaborative scientists studying a wide range of taxa and ecosystem functions. To date over 90 researchers from 23 different institutions have been involved with research projects in the SAFE area. The SAFE Project provides a good example of the benefits of closer collaboration between stakeholders in the development of conservation initiatives and a more sustainable palm oil industry.

\subsection*{Introduction}
\begin{itemize}
	\item Agricultural land has expanded rapidly in the tropics, great money wise, not so great in terms of global diversity.
	\item Southeast asia is particularly bad!
	\item Palm Oil has lots of uses - and new ones are regularly found! (Biofuels etc.) so demand is only set to increase.
	\item Species are found in Oil Palm Plantations (particularly fragments)
	\item however most species found in oil palm are not what would have been there originally and of lower conservation value.
	\item loss in biodiversity is probably directly linked to ecosystem functioning (pollination, pest control, decomposition, carbon sequestration) and loss of resilience/stability
	\item there is a need to understand how landscapes can be managed to maintain biodiversity and also sustainable crop production
	\item What is the optimal size and placement of fragments?
	\item land sharing vs land sparing
	\item areas of natural habitat could provide some functions for agriculture
\end{itemize}

\subsection*{Objectives}
\begin{enumerate}
	\item describe the rationale of SAFE
	\item describe key achievements to date
	\item highlight importance of close collaboration between industry and research
	\item introduce future plans
\end{enumerate}

\subsection*{Materials \& Methods}
\textbf{Rationale}
\begin{itemize}
	\item Multidisciplinary
	\item Inform sustainable land-management practices
	\item has funding for enough time to research temporal aspects
	\item on a scale appropriate for its aims (i.e. Massive!)
	\item researchers from loads of different institutions
\end{itemize}
\textbf{Project Design}
\begin{itemize}
	\item Sampling points across a landscape gradient
	\begin{itemize}
		\item Primary Forest
		\item logged forest to remain forested
		\item logged forest earmarked for conversion
		\item existing oil palm plantation
	\end{itemize}
	\item in each area a range of fragments will be maintained (1ha, 10ha 100ha) in six replicate blocks
\end{itemize}

\subsection*{Results}
\begin{itemize}
	\item lots of initial data on forest quality
	\item definitions of poor/ok/good etc...
\end{itemize}
