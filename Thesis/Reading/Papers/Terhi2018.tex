\section*{Logging disturbance shifts net primary productivity and its allocation in Bornean tropical forests \citep{Riutta2018}}

\subsection*{Abstract}
Tropical forests play a major role in the carbon cycle of the terrestrial biosphere. Recent field studies have provided detailed descriptions of the carbon cycle of mature tropical forests, but logged or secondary forests have received much less attention. Here, we report the first measures of total net primary productivity (NPP) and its allocation along a disturbance gradient from old-growth forests to moderately and heavily logged forests in Malaysian Borneo. We measured the main NPP components (woody, fine root and canopy NPP) in old-growth (n = 6) and logged (n = 5) 1 ha forest plots. Overall, the total NPP did not differ between old-growth and logged forest (13.5 $\pm$ 0.5 and 15.7  $\pm$ 1.5 Mg C ha\textsuperscript{-1}year\textsuperscript{-1} respectively). However, logged forests allocated significantly higher fraction into woody NPP at the expense of the canopy NPP (42\% and 48\% into woody and canopy NPP, respectively, in old-growth forest vs 66\% and 23\% in logged forest). When controlling for local stand structure, NPP in logged forest stands was 41\% higher, and woody NPP was 150\% higher than in old-growth stands with similar basal area, but this was offset by structure effects (higher gap frequency and absence of large trees in logged forest). This pattern was not driven by species turnover: the average woody NPP of all species groups within logged forest (pioneers, nonpioneers, species unique to logged plots and species shared with old-growth plots) was similar. Hence, below a threshold of very heavy disturbance, logged forests can exhibit higher NPP and higher allocation to wood; such shifts in carbon cycling persist for decades after the logging event. Given that the majority of tropical forest biome has experienced some degree of logging, our results demonstrate that logging can cause substantial shifts in carbon production and allocation in tropical forests.

\subsection*{Introduction}
\begin{itemize}
	\item The tropical forest biome plays a dual role in the global carbon cycle
	\begin{itemize}
		\item Carbon sink for approx 40\% of global land C
		\item Approx 90\% of C emissions from land use change
	\end{itemize}
	\item Selective logging is changing the nature of tropical forest
	\begin{itemize}
		\item Moves from slow-growing, conservative, shade tolerant species to fast-growing, light demanding species
		\item In short term it decreases NPP but long term woody biomass accumulation is higher in logged forest as they devote more metabolism to rapid-growth and resource acquisition over maintenance and defence (as old growth does) 
	\end{itemize}
	\item Questions of the study:
	\begin{itemize}
		\item How does allocation of NPP vary along disturbance gradient
		\item Are the high woody growth rates in logged forest a result of increased NPP, shift in allocation to favour woody production or a bit of both.
		\item How much does species turnover (presence of pioneer species) impact change in NPP across disturbance gradient
		\item Contribution of shifts in tree carbon budgets vs changes in density and structure in determining net change of NPP
	\end{itemize} 
\end{itemize}

\subsection*{Materials and Methods}
\begin{itemize}
	\item Study Sites:
	\begin{itemize}
		\item Table 1. has a really good summary...
		\item no clear differences in soil-nutrition and physical properties between sites
	\end{itemize}
	\item NPP estimates
	\begin{itemize}
		\item woody NPP (stems, coarse roots and branches)
		\item canopy NPP (leaves, twigs and reproductive parts)
		\item fine root NPP	
		\item Woody NPP
		\begin{itemize}
			\item sum of stem, coarse root and branch turnover NPP
			\item measured all tress >10cm diameter at 1.3m.
			\item all plots recensused at least once
			\item height estimated visually on first census and then calculated based on diameter-height relationship on subsequent censuses
			\item aboveground woody biomass estimated with allometric equations with diameter, height and wood density (from global wood density database)
		\end{itemize}
		\item Canopy NPP
		\begin{itemize}
			\item litter fall was used as proxy
		\end{itemize}
		\item Fine root NPP
		\begin{itemize}
			\item measured with root in growth cores - at installation all roots were removed and replaced with root free soil...
			\item harvested every 3 months
		\end{itemize}
		\item Missing components of NPP
		\begin{itemize}
			\item things like allocation to mycorrhizae, volatile organic emissions and loss to herbivory are not included and would likely contribute ~13\% of NPP
		\end{itemize}
	\end{itemize}
	\item Data Analysis
	\begin{itemize}
		\item the aim is to quantify the spatial variation in NPP (within and among plots, between old and new-growth forest). So they pooled all temporal replicates...	
	\end{itemize}
\end{itemize}
